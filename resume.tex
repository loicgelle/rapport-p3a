\section*{Résumé} % Pas de numérotation
\addcontentsline{toc}{section}{Résumé} % Ajout dans la table des matières

J'analyse dans ce rapport les différentes solutions qui peuvent être apportées pour doter les assistants de preuve d'interfaces graphiques ergonomiques. En proposant des interfaces utilisables au sein d'applications Web modernes et dynamiques, l'objectif à long terme est à la fois d'aider les utilisateurs à bâtir des preuves avec des assistants interactifs, et de leur permettre de piloter à distance des preuves distribuées.\\

Le travail constitue d'une part en la proposition de solutions concrètes pouvant être implémentées dans des interfaces utilisateur pour assistants de preuve. D'autre part, on étudie, à travers le développement d'un prototype d'interface utilisateur pour l'assistant de preuve Psyche, la faisabilité de greffer ces solutions à un logiciel existant. Le développement du prototype permet d'obtenir une plateforme satisfaisante qui peut servir de base à des tests d'implémentation. Cependant, on met au passage en lumière la difficulté que constitue le développement \textit{a posteriori} d'une interface utilisateur pour des logiciels existants, qui n'ont pas pris en compte cette éventualité pendant leurs développements.
