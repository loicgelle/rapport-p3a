\section{Résultats et réflexions sur le projet}

\subsection{Résultats obtenus}

En ce qui concerne l'étude de solutions d'ergonomie pour les assistants de preuve, le projet aura donné lieu à des idées que je crois pertinentes, mais qui représentent un travail de développement important. Le prototype se borne donc à proposer une utilisation plus agréable et visuelle de l'outil Psyche que son interface actuelle en ligne de commande.\\

A ce sujet, le travail de développement aura permis de soulever la \textbf{difficulté qu'il y a de concevoir une interface graphique pour un logiciel déjà existant}.  Beaucoup de temps a été consacré à un travail d'adaptation du logiciel plus qu'à la conception même de l'interface.\\

L'interface proposée constitue tout de même une base de travail intéressante pour implémenter les fonctionnalités évoquées dans ce rapport. Elle se base sur une association de solutions -- React, Material Design, Redux, Node -- jeunes et en rapide évolution. Ces technologies permettent selon moi des développements plus rapides que dans la plupart des langages établis en ce qui concerne les questions d'interface graphique. Au-delà de cela, l'application développée \textbf{remplit l'objectif d'implémenter le mode interactif de Psyche au sein d'une interface graphique}.

\subsection{Réflexions sur le déroulement du projet}

Ce rapport est également l'occasion de prendre du recul sur la manière dont s'est déroulé le projet. Je peux en particulier souligner \textbf{quelques axes d'améliorations} qui m'auraient permis d'être encore plus productif.\\

Beaucoup de temps a été passé sur des problèmes de développement, et assez peu sur les questions d'ergonomie, du moins durant les réunions de projet. Ces questions n'ont peut-être pas été assez anticipées et ont conduit plusieurs fonctionnalités importantes en termes d'interface utilisateur à être implémentées dans la dernière ligne droite du projet. Il aurait certainement été plus bénéfique de dresser rapidement une \textbf{vision détaillée du prototype souhaité} en termes de fonctionnalités, quitte à ajuster cette vision tout au long du projet.\\

Je pense aussi que l'on aurait pu mieux s'appuyer sur les qualités de chacun au sein de l'équipe. J'ai passé du temps à effectuer un travail d'adaptation au sein de Psyche, alors qu'en sollicitant mieux Stéphane ce temps aurait pu être considérablement réduit. Le gain de temps aurait pu être reversé aux développements et aux réflexions sur l'interface.\\

Les réunions ont été régulières et nombreuses pendant une bonne partie du projet, mais pas forcément très efficaces. Avec du recul, je pense n'avoir \textbf{pas assez défini précisément les fonctionnalités à implémenter} pendant les réunions, ce qui a nui à mon efficacité en dehors. Une fonctionnalité clairement définie et dont la technique de réalisation est choisie et comprise prend beaucoup moins de temps à implémenter.\\

Enfin, sur un plan plus technique, aucun environnement de développement commun n'a été construit. Il y a notamment eu quelques problèmes où l'application s'exécutait parfaitement sur une configuration et pas du tout sur une autre. \textbf{Créer un vrai environnement de développement} -- à l'intérieur d'un container Docker par exemple -- aurait permis de fixer les versions des outils de développement et d'exécution et d'éviter de nombreux problèmes.