\section*{Conclusion}
\addcontentsline{toc}{section}{Conclusion}

Le projet mené a permis de mettre en lumière des difficultés mais aussi des solutions intéressantes, que ce soit au niveau du développement ou de la conception de l'interface utilisateur d'un assistant de preuve.\\

Le développement a montré que le travail d'adaptation d'un outil existant pour le doter d'une interface graphique est non négligeable. Le logiciel Psyche a subi des modifications pour lui procurer une API minimaliste qui lui permettrait de dialoguer avec un front-end écrit en Javascript. Le projet a aussi permis de mettre en lumière des outils très intéressants comme \textit{Js of ocaml}, qui ont rendu le travail de prototypage plus simple.\\

La conception de l'interface graphique pose également de nombreux défis. Des solutions dynamiques et actuelles comme React et Redux ont permis de construire une interface fonctionnelle et qui permet d'utiliser le logiciel Psyche dans un mode interactif et visuel.\\

Des solutions concrètes de fonctionnalités intéressantes en matière d'ergonomie des assistants de preuve ont été proposées. Si le temps de développement nécessaire à l'implémentation de toutes ces fonctionnalités n'a pas permis de les réaliser dans le cadre du projet, il est clair que seuls les tests de ces types d'hypothèses permettront, à terme, le développement d'assistants de preuves interactifs et dotés de réelles interfaces utilisateur.